\documentclass[12pt, a4paper]{article}

\usepackage{fontspec}
\usepackage{geometry}
\usepackage{lastpage}
\usepackage{fancyhdr}
\usepackage[hidelinks]{hyperref}
\usepackage[normalem]{ulem}
\usepackage{soul}
\usepackage{multicol}
\usepackage[dvipsnames]{xcolor}

\geometry{
  top=1cm,
  bottom=2cm,
  left=1cm,
  right=1cm,
  marginparsep=4pt,
  marginparwidth=1cm
}

\definecolor{lgrey}{rgb}{0.8, 0.8, 0.8}

\renewcommand{\headrulewidth}{0pt}
\pagestyle{fancyplain}
\fancyhf{}
\lfoot{\color{lgrey}2020-07-09}
% \rfoot{\color{lgrey}page \thepage\ of \pageref{LastPage}}

\setlength{\parindent}{0pt}
\setlength{\parskip}{0pt}

\usepackage{xunicode}
\defaultfontfeatures{Mapping=tex-text}

\setromanfont{YaleNew}

\begin{document}

\begin{center}
\textbf{MATH/STAT 289: Introduction to Data Science, Fall 2020}
\end{center}

\noindent
\begin{tabular}{ l l }
\textbf{Instructor:} &  \textbf{Taylor Arnold} \\
E-mail: & \texttt{tarnold2@richmond.edu} \\
Website: & \texttt{https://statsmaths.github.io/stat289-f20}
\end{tabular}

\vspace{0.5cm}

\textbf{Description:} \vspace{6pt}

Data science is an interdisciplinary field concerned with drawing knowledge
from data and communicating those results to various audiences. Data
science needs to be learned \textit{by doing} data science. At the end of the semester,
students will have acquired a toolkit of methods, and the knowledge of how to
use them in practice, to address important social, cultural, and scientific
questions with data-driven techniques.

\bigskip

\textbf{Prerequisites:} \vspace{6pt}

The pace of this course assumes that students have had some prior exposure to a programming
language and have taken a course in which statistical techniques were applied to the
analysis of real-world datasets. Prior knowledge of R is useful but not required. As an
alternative starting point for anyone without this background, we suggest students consider
taking MATH209. Several sections of this course are being offered this semester.

\bigskip

\textbf{Website and Software:} \vspace{6pt}

All of the materials and assignments for the course will be posted online on the course's
website. These notes will continue to be available for your reference after the semester has
finished. This course will make use on an online platform called RStudio Cloud. Students
will be asked to sign-up for an account (internet access and a modern browser are required;
no other software is needed).

\bigskip

\textbf{Grades:} \vspace{6pt}

There will be four class projects assigned throughout the semester. Projects will be
distributed with a rubric and graded with a corresponding letter grade. Final grades will
be determined by taking an equal average of the project scores. There is no attendence
requirement nor are there any final examinations. Additionally, in order to pass the class
you are expected to fill out a homework worksheet form before each class meeting
(as applicable). A link and full details are given on the course website.

\bigskip

\textbf{Regarding COVID-19 and Fall 2020:} \vspace{6pt}

There is a very real possibility the Fall 2020 semester will continue to be disrupted by
the ongoing coronavirus pandemic. We should be prepared to address both community-wide
needs as well as accommodations required for specific groups of people. In an effort to
reduce the anxiety of this possibility on everyone involved, the semester's material has
been modified to be amenable to the needs of moving to a fully online and/or asynchronous
format. Generally, the structure and grading structure of the course should remain largely
unchanged in the event of changes to policies regarding residence and instruction. In the
event of further changes, all reasonable requests for deadline extensions and
accommodations will be honored.

\bigskip

\textbf{Additional Class Policies:} \vspace{6pt}

\begin{itemize}\setlength\itemsep{0em}
\item \textbf{Academic honesty:}
Cheating and plagiarism are grave scholarly offenses and potential grounds
for expulsion; they are also a major barrier to your intellectual development.
You are expected to familiarize yourself with the entirety of the
University of Richmond’s Honor Code.
\item \textbf{Pass/Fail/Withdraw/Incomplete:} If you choose to take this
class pass/fail, it is expected that you will achieve a minimum grade of a B- (my project
rubrics do not specify grades lower than this). I am generally willing to allow withdraws
after the deadline, with approval of your Dean, without penalty. However, I typically only
allow a grade of incomplete (I or Y) in cases where at least half of the semester's projects
have been completed.
\end{itemize}

If you have any questions regarding the course policies that are not covered above, or
that you find unclear, please ask for clarification at any point in the semester.

\end{document}
